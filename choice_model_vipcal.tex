% Options for packages loaded elsewhere
\PassOptionsToPackage{unicode}{hyperref}
\PassOptionsToPackage{hyphens}{url}
%
\documentclass[
]{article}
\usepackage{amsmath,amssymb}
\usepackage{lmodern}
\usepackage{iftex}
\ifPDFTeX
  \usepackage[T1]{fontenc}
  \usepackage[utf8]{inputenc}
  \usepackage{textcomp} % provide euro and other symbols
\else % if luatex or xetex
  \usepackage{unicode-math}
  \defaultfontfeatures{Scale=MatchLowercase}
  \defaultfontfeatures[\rmfamily]{Ligatures=TeX,Scale=1}
\fi
% Use upquote if available, for straight quotes in verbatim environments
\IfFileExists{upquote.sty}{\usepackage{upquote}}{}
\IfFileExists{microtype.sty}{% use microtype if available
  \usepackage[]{microtype}
  \UseMicrotypeSet[protrusion]{basicmath} % disable protrusion for tt fonts
}{}
\makeatletter
\@ifundefined{KOMAClassName}{% if non-KOMA class
  \IfFileExists{parskip.sty}{%
    \usepackage{parskip}
  }{% else
    \setlength{\parindent}{0pt}
    \setlength{\parskip}{6pt plus 2pt minus 1pt}}
}{% if KOMA class
  \KOMAoptions{parskip=half}}
\makeatother
\usepackage{xcolor}
\usepackage[margin=1in]{geometry}
\usepackage{color}
\usepackage{fancyvrb}
\newcommand{\VerbBar}{|}
\newcommand{\VERB}{\Verb[commandchars=\\\{\}]}
\DefineVerbatimEnvironment{Highlighting}{Verbatim}{commandchars=\\\{\}}
% Add ',fontsize=\small' for more characters per line
\usepackage{framed}
\definecolor{shadecolor}{RGB}{248,248,248}
\newenvironment{Shaded}{\begin{snugshade}}{\end{snugshade}}
\newcommand{\AlertTok}[1]{\textcolor[rgb]{0.94,0.16,0.16}{#1}}
\newcommand{\AnnotationTok}[1]{\textcolor[rgb]{0.56,0.35,0.01}{\textbf{\textit{#1}}}}
\newcommand{\AttributeTok}[1]{\textcolor[rgb]{0.77,0.63,0.00}{#1}}
\newcommand{\BaseNTok}[1]{\textcolor[rgb]{0.00,0.00,0.81}{#1}}
\newcommand{\BuiltInTok}[1]{#1}
\newcommand{\CharTok}[1]{\textcolor[rgb]{0.31,0.60,0.02}{#1}}
\newcommand{\CommentTok}[1]{\textcolor[rgb]{0.56,0.35,0.01}{\textit{#1}}}
\newcommand{\CommentVarTok}[1]{\textcolor[rgb]{0.56,0.35,0.01}{\textbf{\textit{#1}}}}
\newcommand{\ConstantTok}[1]{\textcolor[rgb]{0.00,0.00,0.00}{#1}}
\newcommand{\ControlFlowTok}[1]{\textcolor[rgb]{0.13,0.29,0.53}{\textbf{#1}}}
\newcommand{\DataTypeTok}[1]{\textcolor[rgb]{0.13,0.29,0.53}{#1}}
\newcommand{\DecValTok}[1]{\textcolor[rgb]{0.00,0.00,0.81}{#1}}
\newcommand{\DocumentationTok}[1]{\textcolor[rgb]{0.56,0.35,0.01}{\textbf{\textit{#1}}}}
\newcommand{\ErrorTok}[1]{\textcolor[rgb]{0.64,0.00,0.00}{\textbf{#1}}}
\newcommand{\ExtensionTok}[1]{#1}
\newcommand{\FloatTok}[1]{\textcolor[rgb]{0.00,0.00,0.81}{#1}}
\newcommand{\FunctionTok}[1]{\textcolor[rgb]{0.00,0.00,0.00}{#1}}
\newcommand{\ImportTok}[1]{#1}
\newcommand{\InformationTok}[1]{\textcolor[rgb]{0.56,0.35,0.01}{\textbf{\textit{#1}}}}
\newcommand{\KeywordTok}[1]{\textcolor[rgb]{0.13,0.29,0.53}{\textbf{#1}}}
\newcommand{\NormalTok}[1]{#1}
\newcommand{\OperatorTok}[1]{\textcolor[rgb]{0.81,0.36,0.00}{\textbf{#1}}}
\newcommand{\OtherTok}[1]{\textcolor[rgb]{0.56,0.35,0.01}{#1}}
\newcommand{\PreprocessorTok}[1]{\textcolor[rgb]{0.56,0.35,0.01}{\textit{#1}}}
\newcommand{\RegionMarkerTok}[1]{#1}
\newcommand{\SpecialCharTok}[1]{\textcolor[rgb]{0.00,0.00,0.00}{#1}}
\newcommand{\SpecialStringTok}[1]{\textcolor[rgb]{0.31,0.60,0.02}{#1}}
\newcommand{\StringTok}[1]{\textcolor[rgb]{0.31,0.60,0.02}{#1}}
\newcommand{\VariableTok}[1]{\textcolor[rgb]{0.00,0.00,0.00}{#1}}
\newcommand{\VerbatimStringTok}[1]{\textcolor[rgb]{0.31,0.60,0.02}{#1}}
\newcommand{\WarningTok}[1]{\textcolor[rgb]{0.56,0.35,0.01}{\textbf{\textit{#1}}}}
\usepackage{graphicx}
\makeatletter
\def\maxwidth{\ifdim\Gin@nat@width>\linewidth\linewidth\else\Gin@nat@width\fi}
\def\maxheight{\ifdim\Gin@nat@height>\textheight\textheight\else\Gin@nat@height\fi}
\makeatother
% Scale images if necessary, so that they will not overflow the page
% margins by default, and it is still possible to overwrite the defaults
% using explicit options in \includegraphics[width, height, ...]{}
\setkeys{Gin}{width=\maxwidth,height=\maxheight,keepaspectratio}
% Set default figure placement to htbp
\makeatletter
\def\fps@figure{htbp}
\makeatother
\setlength{\emergencystretch}{3em} % prevent overfull lines
\providecommand{\tightlist}{%
  \setlength{\itemsep}{0pt}\setlength{\parskip}{0pt}}
\setcounter{secnumdepth}{-\maxdimen} % remove section numbering
\ifLuaTeX
  \usepackage{selnolig}  % disable illegal ligatures
\fi
\IfFileExists{bookmark.sty}{\usepackage{bookmark}}{\usepackage{hyperref}}
\IfFileExists{xurl.sty}{\usepackage{xurl}}{} % add URL line breaks if available
\urlstyle{same} % disable monospaced font for URLs
\hypersetup{
  pdftitle={Choice of VIPCAL Model},
  pdfauthor={Hisham M Shaikh},
  hidelinks,
  pdfcreator={LaTeX via pandoc}}

\title{Choice of VIPCAL Model}
\author{Hisham M Shaikh}
\date{2022-12-21}

\begin{document}
\maketitle

\textbf{1.0 Setting Up the environment}\\
\strut \\
\textbf{1.1 Setting working directory}\\
\texttt{setwd("C:/Users/hisham.shaikh/OneDrive\ -\ UGent/Projects/FCM\_R/ViralProduction\_R"}\strut \\
\strut \\
\textbf{1.2 Installing Libraries}\\

\begin{Shaded}
\begin{Highlighting}[]
\FunctionTok{library}\NormalTok{(}\StringTok{"lme4"}\NormalTok{)}
\end{Highlighting}
\end{Shaded}

\begin{verbatim}
## Loading required package: Matrix
\end{verbatim}

\begin{Shaded}
\begin{Highlighting}[]
\FunctionTok{library}\NormalTok{(}\StringTok{"emmeans"}\NormalTok{)}
\end{Highlighting}
\end{Shaded}

\hfill\break
\textbf{2.0 Importing and wrangling data}\\

\begin{Shaded}
\begin{Highlighting}[]
\FunctionTok{source}\NormalTok{(}\StringTok{"C:/Users/hisham.shaikh/OneDrive {-} UGent/Projects/FCM\_R/ViralProduction\_R/vp\_functions.R"}\NormalTok{, }\AttributeTok{echo=}\ConstantTok{TRUE}\NormalTok{) }\CommentTok{\#functions written to ease manipulation}
\end{Highlighting}
\end{Shaded}

\begin{verbatim}
## -- Attaching packages --------------------------------------- tidyverse 1.3.2 --
## v ggplot2 3.4.0      v purrr   0.3.5 
## v tibble  3.1.8      v dplyr   1.0.10
## v tidyr   1.2.1      v stringr 1.5.0 
## v readr   2.1.3      v forcats 0.5.2 
## -- Conflicts ------------------------------------------ tidyverse_conflicts() --
## x tidyr::expand() masks Matrix::expand()
## x dplyr::filter() masks stats::filter()
## x dplyr::lag()    masks stats::lag()
## x tidyr::pack()   masks Matrix::pack()
## x tidyr::unpack() masks Matrix::unpack()
## 
## Attaching package: 'scales'
## 
## 
## The following object is masked from 'package:purrr':
## 
##     discard
## 
## 
## The following object is masked from 'package:readr':
## 
##     col_factor
\end{verbatim}

\begin{Shaded}
\begin{Highlighting}[]
\NormalTok{NJ1}\OtherTok{\textless{}{-}} \FunctionTok{read.csv}\NormalTok{(}\StringTok{"NJ1.csv"}\NormalTok{)}
\NormalTok{lmer\_data}\OtherTok{\textless{}{-}} \FunctionTok{df\_sr}\NormalTok{(NJ1)}

\CommentTok{\#only calculating for total virus for now.}

\NormalTok{lmer\_data}\OtherTok{\textless{}{-}}\NormalTok{ lmer\_data[lmer\_data}\SpecialCharTok{$}\NormalTok{count }\SpecialCharTok{==}\StringTok{"c\_Viruses"}\NormalTok{,]}

\NormalTok{lmer\_data}\SpecialCharTok{$}\NormalTok{Sample\_Type }\OtherTok{\textless{}{-}} \FunctionTok{as.factor}\NormalTok{(lmer\_data}\SpecialCharTok{$}\NormalTok{Sample\_Type)}
\NormalTok{lmer\_data}\SpecialCharTok{$}\NormalTok{Timepoint2}\OtherTok{\textless{}{-}} \FunctionTok{as.factor}\NormalTok{(lmer\_data}\SpecialCharTok{$}\NormalTok{Timepoint)}

\NormalTok{lmer\_data[lmer\_data}\SpecialCharTok{$}\NormalTok{Sample\_Type }\SpecialCharTok{==} \StringTok{"VPC"}\NormalTok{,]}\SpecialCharTok{$}\NormalTok{Replicate}\OtherTok{\textless{}{-}} \FunctionTok{replace}\NormalTok{(lmer\_data[lmer\_data}\SpecialCharTok{$}\NormalTok{Sample\_Type }\SpecialCharTok{==} \StringTok{"VPC"}\NormalTok{,]}\SpecialCharTok{$}\NormalTok{Replicate, lmer\_data[lmer\_data}\SpecialCharTok{$}\NormalTok{Sample\_Type }\SpecialCharTok{==} \StringTok{"VPC"}\NormalTok{,]}\SpecialCharTok{$}\NormalTok{Replicate }\SpecialCharTok{==} \FunctionTok{c}\NormalTok{(}\StringTok{"1"}\NormalTok{,}\StringTok{"2"}\NormalTok{, }\StringTok{"3"}\NormalTok{), }\FunctionTok{c}\NormalTok{(}\StringTok{"4"}\NormalTok{, }\StringTok{"5"}\NormalTok{, }\StringTok{"6"}\NormalTok{))}

\NormalTok{lmer\_data[}\FunctionTok{order}\NormalTok{(lmer\_data}\SpecialCharTok{$}\NormalTok{Replicate),]}
\end{Highlighting}
\end{Shaded}

\textbf{3.0 Modeling}\\
Adding all possible models\\

\begin{Shaded}
\begin{Highlighting}[]
\CommentTok{\#Linear Models}

\NormalTok{m1}\OtherTok{\textless{}{-}} \FunctionTok{lm}\NormalTok{(}\AttributeTok{data =}\NormalTok{ lmer\_data, }\AttributeTok{formula =}\NormalTok{ value}\SpecialCharTok{\textasciitilde{}}\NormalTok{ Sample\_Type }\SpecialCharTok{+}\NormalTok{ Timepoint2 }\SpecialCharTok{+}\NormalTok{ Replicate)}
\FunctionTok{summary}\NormalTok{(m1)  }\CommentTok{\#all three predictors}
\NormalTok{m2}\OtherTok{\textless{}{-}} \FunctionTok{lm}\NormalTok{(}\AttributeTok{data =}\NormalTok{ lmer\_data, }\AttributeTok{formula =}\NormalTok{ value}\SpecialCharTok{\textasciitilde{}}\NormalTok{ Sample\_Type }\SpecialCharTok{+}\NormalTok{ Timepoint2)}
\FunctionTok{summary}\NormalTok{(m2) }\CommentTok{\#without replicate}
\NormalTok{m3}\OtherTok{\textless{}{-}} \FunctionTok{lm}\NormalTok{(}\AttributeTok{data =}\NormalTok{ lmer\_data, }\AttributeTok{formula =}\NormalTok{ value}\SpecialCharTok{\textasciitilde{}}\NormalTok{ Sample\_Type }\SpecialCharTok{+}\NormalTok{ Timepoint2 }\SpecialCharTok{+}\NormalTok{ Replicate }\SpecialCharTok{+}\NormalTok{ Sample\_Type}\SpecialCharTok{*}\NormalTok{Timepoint2)}
\FunctionTok{summary}\NormalTok{(m3) }\CommentTok{\#with an interaction between sample type and time points}
\NormalTok{m4}\OtherTok{\textless{}{-}} \FunctionTok{lm}\NormalTok{(}\AttributeTok{data =}\NormalTok{ lmer\_data, }\AttributeTok{formula =}\NormalTok{ value}\SpecialCharTok{\textasciitilde{}}\NormalTok{ Sample\_Type }\SpecialCharTok{+}\NormalTok{ Timepoint2 }\SpecialCharTok{+}\NormalTok{  Sample\_Type}\SpecialCharTok{*}\NormalTok{Timepoint2)}
\FunctionTok{summary}\NormalTok{(m4) }\CommentTok{\#with an interaction between sample type and time points and without replicates}

\CommentTok{\#Linear Mixed Effect models}
\CommentTok{\#As we wanted to keep the identity of replicates. We add replicates as the random effect factor}

\NormalTok{m5}\OtherTok{\textless{}{-}} \FunctionTok{lmer}\NormalTok{(}\AttributeTok{data =}\NormalTok{ lmer\_data, }\AttributeTok{formula =}\NormalTok{ value}\SpecialCharTok{\textasciitilde{}}\NormalTok{ Sample\_Type }\SpecialCharTok{+}\NormalTok{ Timepoint2 }\SpecialCharTok{+}\NormalTok{ Sample\_Type}\SpecialCharTok{*}\NormalTok{Timepoint2 }\SpecialCharTok{+}\NormalTok{ (}\DecValTok{1} \SpecialCharTok{|}\NormalTok{ Replicate))}
\FunctionTok{summary}\NormalTok{(m5) }\CommentTok{\#Both predictors and interaction between them }
\NormalTok{m6}\OtherTok{\textless{}{-}} \FunctionTok{lmer}\NormalTok{(}\AttributeTok{data =}\NormalTok{ lmer\_data, }\AttributeTok{formula =}\NormalTok{ value}\SpecialCharTok{\textasciitilde{}}\NormalTok{  Sample\_Type}\SpecialCharTok{*}\NormalTok{Timepoint2 }\SpecialCharTok{+}\NormalTok{ (}\DecValTok{1} \SpecialCharTok{|}\NormalTok{ Replicate))}
\FunctionTok{summary}\NormalTok{(m6) }\CommentTok{\#Only interaction term. We are trying to predict value over a combination of sample type and time points when we keep all points of a replicate together.}
\NormalTok{m7}\OtherTok{\textless{}{-}} \FunctionTok{lmer}\NormalTok{(}\AttributeTok{data =}\NormalTok{ lmer\_data, }\AttributeTok{formula =}\NormalTok{ value}\SpecialCharTok{\textasciitilde{}}\NormalTok{  Sample\_Type}\SpecialCharTok{*}\NormalTok{Timepoint2 }\SpecialCharTok{+}\NormalTok{ (}\DecValTok{1} \SpecialCharTok{+}\NormalTok{ value}\SpecialCharTok{|}\NormalTok{ Replicate))}
\end{Highlighting}
\end{Shaded}

\begin{verbatim}
## boundary (singular) fit: see help('isSingular')
\end{verbatim}

\begin{Shaded}
\begin{Highlighting}[]
\FunctionTok{summary}\NormalTok{(m7) }\CommentTok{\#only interaction term, but where replicates vary with value. This model doesn\textquotesingle{}t make sense as what we\textquotesingle{}re basically saying is: we are trying to predict value over a combination of sample type and time points when replicates vary with value itself. This collapses our model.}
\NormalTok{m8}\OtherTok{\textless{}{-}} \FunctionTok{lmer}\NormalTok{(}\AttributeTok{data =}\NormalTok{ lmer\_data, }\AttributeTok{formula =}\NormalTok{ value}\SpecialCharTok{\textasciitilde{}}\NormalTok{  Sample\_Type}\SpecialCharTok{*}\NormalTok{Timepoint2 }\SpecialCharTok{+}\NormalTok{ (}\DecValTok{1} \SpecialCharTok{+}\NormalTok{ Sample\_Type}\SpecialCharTok{|}\NormalTok{ Replicate))}
\FunctionTok{summary}\NormalTok{(m8) }\CommentTok{\#only interaction term, but where replicates vary with sample type. We are trying to predict value over a combination of sample type and time points when replicates vary with sample\_type.}
\end{Highlighting}
\end{Shaded}

AIC was calculated for each of these models\\

\begin{Shaded}
\begin{Highlighting}[]
\NormalTok{AIC.table }\OtherTok{\textless{}{-}}\NormalTok{ MuMIn}\SpecialCharTok{::}\FunctionTok{model.sel}\NormalTok{(m1, m2, m3, m4, m5, m6, m7, m8)}
\NormalTok{AIC.table }\OtherTok{\textless{}{-}}\NormalTok{ AIC.table[, }\FunctionTok{c}\NormalTok{(}\StringTok{"df"}\NormalTok{, }\StringTok{"logLik"}\NormalTok{, }\StringTok{"AICc"}\NormalTok{, }\StringTok{"delta"}\NormalTok{)]}
\NormalTok{AIC.table}
\end{Highlighting}
\end{Shaded}

\begin{verbatim}
##    df     logLik      AICc    delta
## m7 16  -84.95748  230.5465   0.0000
## m5 14 -362.10354  772.2071 541.6605
## m6 14 -362.10354  772.2071 541.6605
## m8 16 -361.78743  784.2064 553.6599
## m4 13 -526.61560 1095.7767 865.2301
## m3 17 -520.67833 1109.3567 878.8101
## m2  8 -551.35174 1124.0368 893.4903
## m1 12 -550.02427 1137.6138 907.0672
\end{verbatim}

As you can see the best models are m7, m5, m6, and m8 (m7\textgreater{}
m5, m6 \textgreater{} m8). All of them are mixed effect models. m5 and
m6 are identical in terms of their assessment. Therefore, there is no
effect of adding Sample Type and Timepoint as separate predictor
variables.\\
\strut \\
m7 is bogus, as we previously established\\
\strut \\

Yet, let's assess the fit of these models.\\

\emph{Normalized Residuals }

\begin{Shaded}
\begin{Highlighting}[]
\FunctionTok{par}\NormalTok{(}\AttributeTok{mfrow =} \FunctionTok{c}\NormalTok{(}\DecValTok{2}\NormalTok{,}\DecValTok{2}\NormalTok{))}
\ControlFlowTok{for}\NormalTok{( model }\ControlFlowTok{in} \DecValTok{5}\SpecialCharTok{:}\DecValTok{8}\NormalTok{)\{}
\NormalTok{ a}\OtherTok{\textless{}{-}} \FunctionTok{paste0}\NormalTok{(}\StringTok{"m"}\NormalTok{, model)}
 
\FunctionTok{hist}\NormalTok{(}\FunctionTok{resid}\NormalTok{(}\FunctionTok{get}\NormalTok{(a)),}
      \AttributeTok{xlab =} \StringTok{"Normalized Residuals"}\NormalTok{,}
     \AttributeTok{main =} \FunctionTok{paste}\NormalTok{(}\StringTok{"LMEM"}\NormalTok{, a , }\AttributeTok{sep =} \StringTok{"\_"}\NormalTok{)) }
  
\NormalTok{\}}
\end{Highlighting}
\end{Shaded}

\includegraphics{choice_model_vipcal_files/figure-latex/unnamed-chunk-3-1.pdf}

\emph{Normalized Residuals vs Predicted Values plots}

\begin{Shaded}
\begin{Highlighting}[]
\FunctionTok{par}\NormalTok{(}\AttributeTok{mfrow =} \FunctionTok{c}\NormalTok{(}\DecValTok{2}\NormalTok{,}\DecValTok{2}\NormalTok{))}
\ControlFlowTok{for}\NormalTok{( model }\ControlFlowTok{in} \DecValTok{5}\SpecialCharTok{:}\DecValTok{8}\NormalTok{)\{}
\NormalTok{  a}\OtherTok{\textless{}{-}} \FunctionTok{paste0}\NormalTok{(}\StringTok{"m"}\NormalTok{, model)}
   \FunctionTok{plot}\NormalTok{(}\FunctionTok{resid}\NormalTok{(}\FunctionTok{get}\NormalTok{(a)) }\SpecialCharTok{\textasciitilde{}} \FunctionTok{fitted}\NormalTok{(}\FunctionTok{get}\NormalTok{(a)), }
        \AttributeTok{xlab =} \StringTok{"Predicted values"}\NormalTok{, }
        \AttributeTok{ylab =} \StringTok{"Normalized residuals"}\NormalTok{, }
        \AttributeTok{main =} \FunctionTok{paste}\NormalTok{(}\StringTok{"LMEM"}\NormalTok{, a , }\AttributeTok{sep =} \StringTok{"\_"}\NormalTok{))}
   \FunctionTok{abline}\NormalTok{(}\AttributeTok{h =} \DecValTok{0}\NormalTok{, }\AttributeTok{lty =} \DecValTok{2}\NormalTok{)}
  
\NormalTok{\}}
\end{Highlighting}
\end{Shaded}

\includegraphics{choice_model_vipcal_files/figure-latex/unnamed-chunk-4-1.pdf}
m5, m6, m7 are quite similar.\\

Let's plot the predicted vs observed values to asess the models
further\\

\emph{Normalized Residuals vs Replicates plots}

\begin{Shaded}
\begin{Highlighting}[]
\FunctionTok{par}\NormalTok{(}\AttributeTok{mfrow =} \FunctionTok{c}\NormalTok{(}\DecValTok{2}\NormalTok{,}\DecValTok{2}\NormalTok{))}
\ControlFlowTok{for}\NormalTok{( model }\ControlFlowTok{in} \DecValTok{5}\SpecialCharTok{:}\DecValTok{8}\NormalTok{)\{}
\NormalTok{  a}\OtherTok{\textless{}{-}} \FunctionTok{paste0}\NormalTok{(}\StringTok{"m"}\NormalTok{, model)}
  \FunctionTok{print}\NormalTok{(a)}
  \FunctionTok{boxplot}\NormalTok{(}\FunctionTok{resid}\NormalTok{(}\FunctionTok{get}\NormalTok{(a)) }\SpecialCharTok{\textasciitilde{}}\NormalTok{ Replicate, }
          \AttributeTok{data =}\NormalTok{ lmer\_data, }
          \AttributeTok{xlab =} \StringTok{"Replicate"}\NormalTok{,}
          \AttributeTok{ylab =} \StringTok{"Normalized residuals"}\NormalTok{,}
          \AttributeTok{main =} \FunctionTok{paste}\NormalTok{(}\StringTok{"LMEM"}\NormalTok{, a , }\AttributeTok{sep =} \StringTok{"\_"}\NormalTok{))}
 
  \FunctionTok{abline}\NormalTok{(}\AttributeTok{h =} \DecValTok{0}\NormalTok{, }\AttributeTok{lty =} \DecValTok{2}\NormalTok{)}
  
\NormalTok{\}}
\end{Highlighting}
\end{Shaded}

\begin{verbatim}
## [1] "m5"
\end{verbatim}

\begin{verbatim}
## [1] "m6"
\end{verbatim}

\begin{verbatim}
## [1] "m7"
\end{verbatim}

\begin{verbatim}
## [1] "m8"
\end{verbatim}

\includegraphics{choice_model_vipcal_files/figure-latex/unnamed-chunk-5-1.pdf}

\emph{Predicted Values }

\begin{Shaded}
\begin{Highlighting}[]
\FunctionTok{par}\NormalTok{(}\AttributeTok{mfrow =} \FunctionTok{c}\NormalTok{(}\DecValTok{2}\NormalTok{,}\DecValTok{2}\NormalTok{))}
\ControlFlowTok{for}\NormalTok{( model }\ControlFlowTok{in} \DecValTok{5}\SpecialCharTok{:}\DecValTok{8}\NormalTok{)\{}
\NormalTok{ a}\OtherTok{\textless{}{-}} \FunctionTok{paste0}\NormalTok{(}\StringTok{"m"}\NormalTok{, model)}
 
\FunctionTok{hist}\NormalTok{(}\FunctionTok{predict}\NormalTok{(}\FunctionTok{get}\NormalTok{(a)),}
      \AttributeTok{xlab =} \StringTok{"Predicted Values"}\NormalTok{,}
     \AttributeTok{main =} \FunctionTok{paste}\NormalTok{(}\StringTok{"LMEM"}\NormalTok{, a , }\AttributeTok{sep =} \StringTok{"\_"}\NormalTok{)) }
  
\NormalTok{\}}
\end{Highlighting}
\end{Shaded}

\includegraphics{choice_model_vipcal_files/figure-latex/unnamed-chunk-6-1.pdf}

\emph{Predicted vs Observed Values plots}

\begin{Shaded}
\begin{Highlighting}[]
\FunctionTok{par}\NormalTok{(}\AttributeTok{mfrow =} \FunctionTok{c}\NormalTok{(}\DecValTok{2}\NormalTok{,}\DecValTok{2}\NormalTok{))}
\ControlFlowTok{for}\NormalTok{( model }\ControlFlowTok{in} \DecValTok{5}\SpecialCharTok{:}\DecValTok{8}\NormalTok{)\{}
\NormalTok{  a}\OtherTok{\textless{}{-}} \FunctionTok{paste0}\NormalTok{(}\StringTok{"m"}\NormalTok{, model)}
  \FunctionTok{plot}\NormalTok{(}\FunctionTok{predict}\NormalTok{(}\FunctionTok{get}\NormalTok{(a)) }\SpecialCharTok{\textasciitilde{}}\NormalTok{ lmer\_data}\SpecialCharTok{$}\NormalTok{value, }
       \AttributeTok{xlab =} \StringTok{"Observed values"}\NormalTok{, }
       \AttributeTok{ylab =} \StringTok{"Predicted Values"}\NormalTok{, }
       \AttributeTok{main =} \FunctionTok{paste}\NormalTok{(}\StringTok{"LMEM"}\NormalTok{, a , }\AttributeTok{sep =} \StringTok{"\_"}\NormalTok{))}
  \FunctionTok{abline}\NormalTok{(}\AttributeTok{a=} \DecValTok{0}\NormalTok{, }\AttributeTok{b =} \DecValTok{1}\NormalTok{)}
  
\NormalTok{\}}
\end{Highlighting}
\end{Shaded}

\includegraphics{choice_model_vipcal_files/figure-latex/unnamed-chunk-7-1.pdf}

The residuals of m7 are extremely low. We could take either, m5, m6, or
m8. As m6 and m5 are identical, we prefer m6 as it is simpler.\\
AIC values suggest m6 is slightly better than m8.\\

\textbf{\emph{Our winner is M6!}}

\end{document}
